\section{Fazit und Ausblick (Alle)}
Vergleicht man die aufgeführten Beispiele und dessen Ergebnisse miteinander, so kommt man zu dem Schluss, dass neuronale Netze durchaus Anwendung im Bereich pattern recognition finden. Auch wurde einerseits gezeigt, dass sowohl CNNs als auch Resnets für Speech recognition eingesetzt werden. Andererseits wurde aufgezeigt, dass sich CNNs auch im Bereich emotion recognition etabliert haben und wie in \cite{wildemotionrec} bewiesen, sogar am Besten für facial emotion expression recognition in unkontrollierten Umgebungen geeignet sind. Ebenfalls wurden verschiedene Ansätze für den Einsatz von CNNs gegeben, was zeigt, dass CNNs auf unterschiedliche Weise fungieren können, um diverse Aufgaben zu erledigen. Auch wird bezogen auf die aufgeführten Beispiele deutlich, dass die genannten CNNs mitunter zu den momentan besten Lösungen für die Aufgaben Speech- sowie emotion recognition gehören. Erwähnenswert ist außerdem, dass unter aktuelle Grenzen u.a. die Fehlerrate für das Erkennen von Sprachsignalen und Emotionen gehört, welche momentan noch keine einhundert prozentige Zuordnung erreicht, Nebeneffekte wie die Überanpassung von Neuronen aufgrund zu tiefer Spezialisierung und das Zusammenführen von Wut und Neutralität in der Disziplin emotion recognition aufgrund der nicht zu identifizierbaren Unterschiede. Desweiteren wurde gezeigt, dass NNs bei korrekter Konfiguration, bessere Ergebnisse im Bereich SCE und SEE aufweisen als herkömmliche mathematische Modelle. Der Einsatz von NNS bei SCE und SEE ist noch nicht vollständig sinnvoll, da die Anwendung situations- und projektabhängig ist. Auch sind NNs dann nicht sinnvoll, wenn noch nicht genügend qualitative Trainingsdaten vorliegen, oder die Zeit für das Training der NN nicht gegeben ist.
U.a. hat sich auch ergeben, dass NNs gute Ergebnisse auf dem Gebiet der Softwarequalität erzielen. Dabei wurden mit NNs automatisierte Tests erstellt, welche sowohl Zeit als auch Arbeitsaufwand einsparen konnten. Zusätzlich bieten NNs im Bereich SQ eindeutig bessere Vorhersagen von SQ, was die Arbeit an Softwareprojekten nachhaltig prägt. Aus den genannten Ergebnissen können wir bestätigen, dass neuronale Netze in der Softwaretechnik eingesetzt werden.\\

NEU NEU NEU\\

Ravell START\\
Die aufgeführten Versuche haben gezeigt, dass neuronale Netze im Bereich der Mustererkennung vielerlei Anwendung finden. Vorallem die CNN werden in beiderlei Disziplinen in Verbindung mit diversen Methoden und Vorgehensweisen eingesetzt. Aktuelle Grenzen liegen momentan in der Skalierbarkeit der Schichten, sowie in der Qualität der Trainings- und Testdaten. Zukünftig wären neue Kombinationen oder sogar neue Arten von neuronalen Netzen möglich, um noch bessere und vorallem schnellere Ergebnisse zu erzielen, ohne immenses vorheriges Training oder Konfigurieren.\\
RAVELL ENDE\\
ROBIN START\\
Dieses Kapitel hat verschiedene Möglichkeiten aufgezeigt, wie NNs in den wichtigen Bereichen der SCE und SEE innerhalb der SWT eingesetzt werden. Wenn für gewisse Projekte passende NNs ausgewählt und korrekt konfiguriert werden, liefern NNs bessere Ergebnisse als reine mathematische Modelle. Vor allem in der agilen Entwicklung lösen NNs viele Probleme anderer Ansätze, auch in Verbindung und als Erweiterung der anderen Modelle.
Dennoch sind sie nicht in ausschließlich allen Fällen die beste Lösung. Es existiert noch kein NN, welches eine Universallösung für alle Softwareschätzungen darstellt. Für manche Projekte bietet sich die Lösung mit NNs mehr an als für andere, vor allem da die optimale Lösung noch nicht während der Planung bekannt ist. Vor allem in Umgebungen mit wenigen Daten oder wenig Zeit sind NNs noch nicht praktikabel, da ein passendes NN zuerst entwickelt und trainiert werden muss.\\
ROBIN ENDE\\
KONSTANTIN START\\
Betrachtet man die genannten Beispiele für Einsatzmöglichkeiten von NN's, stellt man fest, dass es Anwendungsfälle gibt in denen sich NN's als sehr geeignet und effizient erweisen. In diesem Abschnitt sehen wir wie Softwaretests automatisch generiert werden können, um Zeit und Arbeitsaufwand zu sparen. Darüber hinaus erkennen wir die Vorteile von NN's zur Vorhersage und Bestimmung von SQ. Gerade das Vorhersagen von SQ, um Rückschlüsse auf Ursachen ziehen zu können, bietet eine große Chance, um die Arbeit an Softwareprojekten nachhaltig zu prägen und zu verbessern. Darum kann bestätigt werden, dass NN's im Bereich der Softwareentwicklung eingesetzt werden können, um die Qualität zu verbessern.\\
KONSTANTIN ENDE\\
