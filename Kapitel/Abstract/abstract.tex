\section*{Abstract (Alle)}
Künstliche neuronale Netze (KNNs) haben in den letzten zwanzig Jahren ein exorbitantes Interesse vielerlei Bereiche erweckt. Ihre Entwicklung geht bis in die fünfziger Jahre zurück und erfährt seitdem immer wieder neue Fortschritte. Dies liegt vor allem daran, dass sie sich weiterentwickeln und dazulernen können und somit auch für zukünftige Themen und Aufgaben geeignet sind. Besonders im Bereich der Softwaretechnik sind künstliche neuronale Netze heute nicht mehr wegzudenken. In dieser Arbeit werden die Disziplinen Mustererkennung, Kosten- und Aufwandsschätzung sowie Qualitätsmanagement näher betrachtet und Beispiele aufgezeigt, um den Einsatz von KNNs sowie die aktuell bekannten Grenzen zu erläutern. Dafür werden relevante Begriffe und Methoden beschrieben. Die verschiedenen Arten von neuronalen Netzen und deren Vor- sowie Nachteile werden in dieser Arbeit aufgezeigt. Das Ergebnis der Arbeit ist, dass sich neuronale Netze fest in der Softwaretechnik etabliert haben, und jede aufgeführte Art in jeder der beschriebenen Disziplinen sowohl eindeutige Verbesserungen als auch Defizite gegenüber klassischen Methoden aufweist. Daraus resultiert, dass es noch Verbesserungsbedarf sowie -möglichkeiten für die Entwicklung von künstlichen neuronalen Netzen gibt.