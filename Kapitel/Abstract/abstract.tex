\section*{Abstract (Alle)}
Künstliche neuronale Netze haben in den letzten zwanzig Jahren ein exorbitantes Interesse für vielerlei Bereiche erweckt. Ihre Entwicklung geht bis in die fünfziger Jahre zurück und erfährt seitdem immer wieder neue Fortschritte. Dies liegt vor allem daran, dass sie sich weiterentwickeln und dazulernen können und somit auch für zukünftige Themen und Aufgaben geeignet sind. Besonders im Bereich der Softwaretechnik sind künstliche neuronale Netze heute nicht mehr wegzudenken. In dieser Arbeit wird näher auf die Disziplinen Mustererkennung, Kosten- und Aufwandsschätzung sowie Qualitätsmanagement eingegangen und Beispiele aufgezeigt, um die Frage zu beantworten, wie neuronale Netze in der Softwaretechnik eingesetzt werden können und wo die aktuell bekannten Grenzen sind. Dafür werden relevante Begriffe und Methoden erläutert sowie handfeste Beweise referenziert. Auch vergleichen wir in unserer Arbeit die verschiedenen Arten von neuronalen Netzen und zeigen Vor- sowie Nachteile auf. Das Ergebnis unserer Arbeit ist, dass sich neuronale Netze fest in der Softwaretechnik etabliert haben, und jede aufgeführte Art in jeder der beschriebenen Disziplinen sowohl eindeutige Verbesserungen als auch Defizite gegenüber klassischen Methoden aufweist. Daraus resultiert, dass es noch Verbesserungsbedarf sowie -möglichkeiten für die Entwicklung von künstlichen neuronalen Netzen gibt. In Zukunft könnten sich noch mehr Gebiete erschließen, welche mit neuronalen Netzen gekoppelt werden, um schnellere und bessere Ergebnisse zu erzielen. Vorstellbar ist auch, dass sich komplett neue Arten von neuronalen Netzen bilden, da selbst heute schon diverse Typen kombiniert werden, um zu neuen Erkenntnissen zu gelangen.