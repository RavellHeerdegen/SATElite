\section{Einleitung (Alle)}
Diese Arbeit soll beantworten, ob neuronale Netze (NNs) in der Softwaretechnik (SWT) eingesetzt werden können und wo die Grenzen der Anwendungen liegen. Die Arbeit soll keine Umfrage sein, ob NNs in der Realität eingesetzt werden. Vielmehr werden drei Themengebiete vorgestellt, in denen Forscher NNs testeten und einsetzten. Unser Ziel ist es, anhand der Themengebiete und konkreter Beispiele herauszufinden, ob und wie NNs in der SWT eingesetzt werden können, welche Probleme sie lösen und an welche Grenzen sie stoßen. Sobald NNs bessere Ergebnisse als herkömmliche Methoden erzielen, ohne den Aufwand oder die Kosten signifikant zu erhöhen, gehen wir davon aus, dass ihr Einsatz in einem bestimmten Gebiet möglich ist.
Künstliche neuronale Netzwerke, auch artifical neural networks(ANNs) oder nur neural networks(NNs) basieren auf der Funktion des biologischen Nervensystems. Informationen werden durch eine Vielzahl miteinander verknüpfter parallel laufender Einheiten verarbeitet. Die daraus entstehenden Systeme trainieren sich teilweise selbst, um schnellere Entscheidungen treffen und Fehler vermeiden zu können.\cite{technology} 
NNs lassen sich in Kategorien unterteilen. Bekannte sind convolutional neural networks (CNNs), deep neural networks (DNNs) und multilayer perception(MLP)-NNs.
In dieser Arbeit wird auf verschiedene NN-Arten eingegangen. Convolutional neural networks (CNNs) sind feed-forward NNs. Das bedeutet, dass Signale durch die verschiedenen Schichten (Layer) des NNs gegeben werden..\cite{usingcnn} Residual neural networks (Resnets) gehören zur Kategorie der deep neural networks (DNNs). Jeder Layer des Resnets besitzt mehrere trainierbare Neuronen. Das Besondere ist der Dropout, welcher durch zufälliges Ausschalten von Neuronen Überanpassungen vermeidet.\cite{residualnn} Multilayer perceptrons (mehrschichtige neuronale Netze) benutzen u.a. Back propagation, um Parameter für künstliche neuronale Netze zu trainieren. Back propagation dient dem Trainieren von Layern, genauer dessen Neuronen, um auf entstandene Fehler aufmerksam zu machen, welche rückwärts durch das NN gegeben werden, um die sogenannten shared weights neu auszurichten.\cite{usingcnn} Das Hidden Markov Model (HMM) ist bekannt für seinen Nutzen im Bereich der speech recognition. HMM ist in der Lage Wörter zu modellieren, welche daraufhin einem bestimmten Kontext zugeordnet werden können. Der HMM-Algorithmus gehört zu den unsupervised learning techniques, da er Parameter unkontrolliert schätzt ohne zusätzliche Kennzeichnungen durch Menschen zu benötigen. \cite{hmm} Ein MLP-NN weißt während des Trainings jedem Neuron eine Gewichtung zu, um den Einfluss zwischen Input und Output zu maximieren.\cite{Khalifelu2012} Radial Basis Function-NNs (RBF) werden in den Gebieten der pattern recognition und Funktionsapproximation eingesetzt. Im ersten der folgenden Kapitel wird die Disziplin pattern recognition und dazugehörige Verfahren und Möglichkeiten näher beschrieben. Anschließend wird der Einsatz von NNs im Gebiet cost and effort estimation erläutert und damit verbundene Methoden und Praktiken aufgezeigt. Das letzte Kapitel befasst sich mit neuronalen Netzen im Bereich quality management und gibt einen Einblick auf verwendete Mittel und Vorgehensweisen, welche in dem genannten Bereich notwendig und nützlich sind. Letztendlich folgt ein Fazit, welches die Ergebnisse aus allen Kapiteln vereint, vergleicht und bewertet.