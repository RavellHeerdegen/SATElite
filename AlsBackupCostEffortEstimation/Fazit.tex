\section{Fazit}
Hat sich herausgestellt, dass NNs im Bereich SCE/SEE eignen, da sie in manchen Projekten bessere Schätzungen erreichten als mathematische Methoden.
Ergebnis: Viele NNs sind aufgrund Menschlicher beschränktheit, auf der anderen Seite aufgrund des Forschungsstandes nicht universell eingesetzt werden können. Außerdem ist der Soft Computing Ansatz nicht für alle Arten von Projekten geeignet. Qualität der Daten - viele sind nur Subjektiv und damit Fehleranfällig, Quantität der Daten - nicht genug informationen vorhanden, in zu frühen Phasen der Planung benötigt, zu wenige vorige Projekte als referenz.
Ausblick: Gut geeignet im wachsenden Agilen Umfeld; Mit einer großen Menge an Daten und perfekt passenden NNs könnte eine universelle Lösung gefunden werden.