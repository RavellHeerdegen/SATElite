\section{Softwarequalität (Konstantin Rosenberg)} \label{SQ}
"Softwarequalität ist die Gesamtheit der Merkmale und Merkmalswerte eines Softwareprodukts, die sich auf dessen Eignung beziehen,
festgelegte oder vorausgesetzte
Erfordernisse zu erfüllen"~\cite[p.~20]{Franz2015}.
F. Klaus zitiert in~\cite[p.~20]{Franz2015} die Norm ISO/IEC 25000, dabei unterscheidet er zwischen funktionalen und nicht-funktionalen Merkmalen von Softwarequalität(SQ).
Funktionale Merkmale beziehen sich auf die reine Funktionalität einer Software. Die nicht-funktionalen Merkmale beschreiben die Zuverlässigkeit, Benutzbarkeit, Effizienz, Anpassungsfähigkeit und Übertragbarkeit eines Softwareprodukts.
Um SQ zu erreichen gibt es verschiedene Ansätze. In Kontrast stehen dabei analytische (AQS) und konstruktive Qualitätssicherungsmaßnahmen (KQS). AQS beziehen sich auf das Vermeiden von Fehlern im Vorfeld. KQS hingegen sollen nach Abschluss der Arbeit Fehler aufdecken.~\cite[p.~29]{Franz2015} 
Um SQ konstruktiv zu sichern, schlägt F. Klaus das Einführen von Testmethoden, Testwerkzeugen und die Schulung der Mitarbeiter als Maßnahmen vor. Für eine analytische Sicherung der Qualität nennt er das Überprüfen von Tests und Dokumenten.
Dieser Abschnitt stellt Verfahren vor, um die genannten Maßnahmen mit NN's umzusetzen und beantwortet die Frage, nach Einsatzmöglichkeiten von NN's in der Qualitätssicherung von Softwareprodukten.

\subsection{Testabdeckung}
Die Testabdeckung bezeichnet in der Softwaretechnik die Relation, der durch Tests abgedeckter Ereignisse und der Menge der insgesamt möglichen Fälle. Eine komplette Testabdeckung ist praktisch nicht möglich. Darum werden zur Feststellung einer ausreichenden Testabdeckung Kriterien angelegt.
Kriterien sind z.B die Menge der abgedeckten Befehle oder die Zahl der gesicherten Methoden.~\cite{Antinyan2018}

\subsection{Software Metriken}
Softwaremetriken(SM) dienen in der Softwareentwicklung als Ansatzpunkt zur Bestimmung von Qualität. In einer SM werden Softwareeigenschaften Zahlen zugeordnet. Anhand dieser Zahlen lassen sich Rückschlüsse auf die Qualität der Software ziehen. Aspekten einer Softwaremetrik können z.B die Zahl der Verzweigungen im Quellcode oder die Komplexität der Vererbungshierarchie sein.~\cite[p.~3]{Committee1998}

\subsection{Automatisiertes Testen durch neural networks}
Softwaretests haben die Aufgabe zu erkennen, ob Software den qualitativen Ansprüchen genügt und um Fehler zu finden und zu beheben.~\cite{Wu2008}
L. Wu et. al. beschreibt in ~\cite{Wu2008} ein Verfahren, um mit BPNN's automatisiert funktionale Tests zu erstellen. 
Bei diesem Vorgehen gibt es eine Reihe an Akteuren. Ein bestehendes System oder Programm. Außerdem die Spezifikation der Funktionen, einen Testgenerator, eine Testdatenbank und das BPNN. Der Testgenerator erzeugt zufällige Input-Variablen, passend zu den festgelegten Systemspezifikationen. Die Variablen werden in der Testdatenbank abgespeichert. Das Softwaresystem erzeugt Erwartungswerte. Als Input für das BPNN dienen die erzeugten Testwerte. Das NN verarbeitet die Eingabewerte und generiert Ausgaben, die gegen den Erwartungswert geprüft werden.
Mit diesem Training nähert sich das NN dem entsprechenden Erwartungswert an.
Sobald die Ausgabewerte einen entsprechenden Toleranzwert erreichen, kann der Input der einzelnen Knoten des NN verwendet werden, um Tests abzuleiten. Der Vorgang kann wiederholt werden, bis die Testabdeckung zufriedenstellend ist. Anhand eines Versuchs werden in~\cite{Wu2008} Tests für ein Softwaresystem erzeugt. Dies zeigt die Benutzbarkeit und Effizienz der Methode.
\noindent In ~\cite{Majma2014} beschreiben M. Negar und S. Morteza ebenfalls einen Ansatz, der die Effektivität von NN's deutlich macht. In der Methode werden zwei NN's verwendet. Eines davon dient dem white box testing, das andere dem black box testing. Das NN für das White box testing kann Testfälle für ihm bekannten Code erzeugen. Das zweite NN ermöglicht das finden von Fehlern, ohne das es den Code  kennt. Beide Varianten ergänzen sich für eine bessere Testabdeckung. Mit diesem Vorgehen lassen sich nach ~\cite{Majma2014} haltbare Softwaretests produzieren, sowie Zeit und Aufwand einsparen.

\subsection{Evaluation von Softwarequalität}
Es bietet Vorteile den Stand von Software, im Bezug auf Qualität, zu kennen. Dadurch können Vorbereitungen gtroffen- und früh Mängel aufgedeckt und behoben werden~\cite{Pomorova2013}. Deshalb stellen O. Pomorova et. al. in~\cite{Pomorova2013} eine Methode vor, um mit NN's die Softwarequalität in einer Projektphase zu evaluieren. Als Input für das NN dienen SM's. Davon enthält eine SM geschätzte, die andere die tatsächlichen Werte.
Durch die Ausgabe des NN's wird die evaluierte Softwarequalität beschrieben oder eine Vorhersage über die zukünftige Komplexität des Projekts gemacht. 
In ~\cite{Pomorova2013} wird ein Versuch beschrieben in dem vier verschiedene Softwareprojekte mit dieser Methode eingeschätzt wurden. Dabei erzielte das System genaue Ergebnisse. Es stellte sich jedoch auch heraus, dass eine unzulängliche Anzahl von Bewertungskriterien zu ungenauen Ergebnissen führen kann.
Wenn die Faktoren gegeben sind, bietet dieses Verfahren aber eine genaue Möglichkeit um SQ erkennbar zu machen und dadurch Rückschlüsse auf die Entwicklungsphase zu ziehen.

\subsection{Vorhersage von Softwarequalität}
Durch eine Vorhersage der SQ können Defizite frühzeitig erkannt und verbessert werden. Mitarbeiter können sich besser auf bevorstehende Probleme vorbereiten und erhalten die Möglichkeit, Zusammenhänge zwischen getroffenen Entscheidungen und SQ zu erkennen.~\cite{Peng2009} In ~\cite{Peng2009} wird deshalb ein Vorgehen beschrieben, um eine Vorhersage zu SQ in objektorientierten Systemen zu treffen. Dabei werden SQ's und NN's verwendet, um Vorhersagen zu machen. Die Metriken werden über ein sog. fuzzy neural network verarbeitet. Fuzzy neural networks können auch mit wenig Input genaue Ergebnisse generieren.
Vorteile dieser Variante zu herkömmlichen Ansätzen, zum Vorhersagen von SQ werden in einem Beispiel in ~\cite{Peng2009} deutlich. Dabei zeigt sich, dass die Zusammenhänge zwischen SQ und beeinflussenden Faktoren zu erkennen sind. Außerdem kommt das Verfahren mit einer Vielzahl an Dateiformaten aus.
