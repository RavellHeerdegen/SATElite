\section{Fazit und Ausblick (Alle)}
Vergleicht man die aufgeführten Beispiele und dessen Ergebnisse miteinander, so kommt man zu dem Schluss, dass neuronale Netze durchaus Anwendung im Bereich pattern recognition finden. Auch wurde einerseits gezeigt, dass sowohl CNNs als auch Resnets für Speech recognition eingesetzt werden. Andererseits wurde aufgezeigt, dass sich CNNs auch im Bereich emotion recognition etabliert haben und wie in \cite{wildemotionrec} bewiesen, sogar am Besten für facial emotion expression recognition in unkontrollierten Umgebungen geeignet sind. Ebenfalls wurden verschiedene Ansätze für den Einsatz von CNNs gegeben, was zeigt, dass CNNs auf unterschiedliche Weise fungieren können, um diverse Aufgaben zu erledigen. Auch wird bezogen auf die aufgeführten Beispiele deutlich, dass die genannten CNNs mitunter zu den momentan besten Lösungen für die Aufgaben Speech- sowie emotion recognition gehören. Erwähnenswert ist außerdem, dass unter aktuelle Grenzen u.a. die Fehlerrate für das Erkennen von Sprachsignalen und Emotionen gehört, welche momentan noch keine einhundert prozentige Zuordnung erreicht, Nebeneffekte wie die Überanpassung von Neuronen aufgrund zu tiefer Spezialisierung und das Zusammenführen von Wut und Neutralität in der Disziplin emotion recognition aufgrund der nicht zu identifizierbaren Unterschiede. Desweiteren wurde gezeigt, dass NNs bei korrekter Konfiguration, bessere Ergebnisse im Bereich SCE und SEE aufweisen als herkömmliche mathematische Modelle. Der Einsatz von NNS bei SCE und SEE ist noch nicht vollständig sinnvoll, da die Anwendung situations- und projektabhängig ist. Auch sind NNs dann nicht sinnvoll, wenn noch nicht genügend qualitative Trainingsdaten vorliegen, oder die Zeit für das Training der NN nicht gegeben ist.
U.a. hat sich auch ergeben, dass NNs gute Ergebnisse auf dem Gebiet der Softwarequalität erzielen. Dabei wurden mit NNs automatisierte Tests erstellt, welche sowohl Zeit als auch Arbeitsaufwand einsparen konnten. Zusätzlich bieten NNs im Bereich SQ eindeutig bessere Vorhersagen von SQ, was die Arbeit an Softwareprojekten nachhaltig prägt. Aus den genannten Ergebnissen können wir bestätigen, dass neuronale Netze in der Softwaretechnik eingesetzt werden.