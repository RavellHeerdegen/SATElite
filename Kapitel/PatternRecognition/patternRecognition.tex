\section{Mustererkennung (Ravell Heerdegen)}
Pattern recognition zeichnet sich durch die Möglichkeit aus, Klassifizierungen durchzuführen, welche entweder überwacht oder unüberwacht, also automatisch und ohne menschliche Hilfe ablaufen \cite{svmgmm}. Im Bereich pattern recognition haben sich in den letzten Jahren besonders neuronale Netze als enormer Fortschritt herausgestellt. Mit neuronalen Netzen werden u.a. Muster erkannt und zugeordnet, Umgebungen analysiert, z.B. im Bereich emotion recognition, Merkmale erkannt und spezifischen Klassen zugeordnet oder auch Bewertungen von Performanz durchgeführt.\cite{patternrec}
Im folgenden Kapitel werden zwei Disziplinen bezüglich pattern recognition beschrieben. Der erste Abschnitt befasst sich mit der Disziplin speech recognition. Dabei gehe ich auf die Definition von speech recognition ein und zeige anschließend diverse Beispiele für den Einsatz von neuronalen Netzen in selbiger Disziplin auf.
Im zweiten Abschnitt folgt die Disziplin emotion recognition sowie einige Beispiele für den Einsatz von neuronalen Netzen in der selbigen Disziplin. Abschließend folgt ein Fazit, welches den Inhalt des Kapitels kurz zusammenfasst und die Ergebnisse reflektiert.